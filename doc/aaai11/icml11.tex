 \documentclass{article}
%% For figures
\usepackage{graphicx} % more modern
%\usepackage{epsfig} % less modern
%\usepackage{subfigure} 

% For citations
\usepackage{natbib}
\usepackage{bibspacing}
%\setlength{\bibspacing}{1.1\baselineskip}
 
   %\usepackage[linesnumbered]{algorithm2e}
   
% For algorithms
\usepackage{algorithm}
\usepackage{algorithmic}
 
% As of 2010, we use the hyperref package to produce hyperlinks in the
% resulting PDF.  If this breaks your system, please commend out the
% following usepackage line and replace \usepackage{icml2011} with
 %\usepackage[nohyperref]{icml2011}
\usepackage{hyperref}  
 \usepackage{amsmath} 
% Packages hyperref and algorithmic misbehave sometimes.  We can fix
% this with the following command.
%\newcommand{\theHalgorithm}{\arabic{algorithm}} 

% Employ the following version of the ``usepackage'' statement for
% submitting the draft version of the paper for review.  This will set
% the note in the first column to ``Under review.  Do not distribute.''
%\usepackage{icml2011} 
% Employ this version of the ``usepackage'' statement after the paper has
% been accepted, when creating the final version.  This will set the
% note in the first column to ``Appearing in''
\usepackage[accepted]{icml2011}  
 
%\def\citep#1{\cite{#1}}
%\def\citet#1{\citeauthor{#1} (\citeyear{#1})}
%\def\citet#1{\cite{#1}}

% The \icmltitle you define below is probably too long as a header.
% Therefore, a short form for the running title is supplied here:
\icmltitlerunning{Goal-Directed Knowledge Acquisition}
 
\begin{document} 

\twocolumn[
\icmltitle{Goal-Directed Knowledge Acquisition}

% It is OKAY to include author information, even for blind
% submissions: the style file will automatically remove it for you
% unless you've provided the [accepted] option to the icml2011
% package.
\icmlauthor{Christopher Weber}{christopherweber@hotmail.com} 
%\icmladdress{Utah State University, 4205 Old Main Hill, Logan, UT 84322}
\icmlauthor{Daniel Morwood}{danmorwood@gmail.com}
%\icmladdress{Utah State University, 4205 Old Main Hill, Logan, UT 84322}
\icmlauthor{Daniel Bryce}{daniel.bryce@usu.edu}
\icmladdress{Utah State University, 4205 Old Main Hill, Logan, UT 84322}

% You may provide any keywords that you 
% find helpful for describing your paper; these are used to populate 
% the "keywords" metadata in the PDF but will not be shown in the document
\icmlkeywords{planning}

\vskip 0.3in
]

%\renewcommand{\baselinestretch}{.98}

% \documentclass{article}
% 
% \usepackage{aaai}
% \usepackage{amsmath}
% \usepackage[linesnumbered,ruled,vlined]{algorithm2e}
% 
 \def\und#1{\noindent{\bf #1}:}
% \def\citep#1{\cite{#1}}
% \def\citet#1{\citeauthor{#1} (\citeyear{#1})}
% \def\FFRISKY{{\tt DeFAULT}}
 \def\default{{\tt DeFAULT}}
 \def\goalie{{\tt Goalie}}
% 
% 
% 
% 
\newenvironment{packed_enum}{
\begin{enumerate}
  \setlength{\itemsep}{1pt}
  \setlength{\parskip}{0pt}
  \setlength{\parsep}{0pt}
}{\end{enumerate}}

\newenvironment{packed_itemize}{
\begin{itemize}
  \setlength{\itemsep}{1pt}
  \setlength{\parskip}{0pt}
  \setlength{\parsep}{0pt}
}{\end{itemize}}


% 
% \title{Goal-Directed Knowledge Acquisition}
% \author{Paper ID: 441}
% 
% \begin{document}
% \nocopyright
% \maketitle

\begin{abstract}
Agents with incomplete knowledge of their actions can either plan around the
incompleteness, ask questions of a domain expert, or learn through trial and
error.  We present and evaluate several approaches to formulating plans under
incomplete information, using the plans to identify relevant (goal-directed)
questions, and interleaving acting, planning, and question asking.  

We show that goal-directed knowledge acquisition leads to fewer questions and
lower overall planning and re-planning time than naive approaches that ask many
questions, or learn by trial and error.  Moreover, we show that prioritizing
questions based on plan failure diagnoses requires less questions on average.
\end{abstract}

\section{Introduction}

Knowledge engineering \citep{ickeps09} and machine learning
\citep{arms,DBLP:conf/aaai/OatesC96} have been applied to constructing
representations for planning, but pose intensive human and/or data requirements,
only to leave a potential mismatch between the environment and model
\citep{modellite}.  Recently, \citet{bryce-icaps11} showed that instead of
placing effort upon making domains complete it is possible to plan with
incomplete knowledge of an agent's action descriptions (i.e., plan around the
incompleteness).  Agents executing such robust plans fail and re-plan less often,
achieving their goals in less overall time than agents that ignore
incompleteness when planning \citep{DBLP:conf/aips/ChangA06}.  While
\citet{bryce-icaps11} demonstrate that planning in incomplete domains can
help agents learn about domains, they ignore cases where domain experts are
available to help engineer the agent's knowledge.  We extend the work of
\citet{bryce-icaps11} (including their \default{} planner) to consider agents
that can query a domain expert, as in instructable computing \citep{mable}, but carefully select their questions for
knowledge acquisition (KA) in a goal-directed fashion to reduce domain expert
fatigue.

Selecting questions is a problem that has been studied in problems such as
preference elicitation \citep{DBLP:conf/aaai/Boutilier02}, machine learning
\citep{AICPub1812:2011}, and model-based diagnosis
\citep{deKleer:1992:CDS:140524.140531}.  KA in planning with
incomplete domain knowledge is unique in that plans have rich causal structure
that makes questions highly coupled, frequent re-planning can change which
questions are relevant, and planning relaxations provide opportunities for
selecting relevant questions at a fraction of the cost of analyzing  plans.  

Our approach to formulating questions relies on planning with incomplete
information, deriving a plan failure explanation (set of diagnoses), and ranking
questions based upon these diagnoses so that questions are goal-directed.  Upon
finding a plan that will provably succeed, our agent executes the plan to
achieve its goals.  We compare our methods with simpler strategies that either
ask questions about all possible incomplete domain features prior to planning, or ask no questions and
learn by trial and error.

This work investigates several issues, including whether i)
planning with incompleteness is helpful when agents have the ability to ask questions, ii) how
plans can effectively support goal-directed knowledge acquisition, and iii)
relaxed plans provide the same benefit of actual plans in guiding knowledge
acquisition.  We find that, indeed, i) planning with incompleteness leads to
fewer required questions and lower overall time to solve a task, ii) ranking
goal-directed questions by a measure of their one-step Shannon entropy is the
most effective method to knowledge acquisition, and iii) relaxed plans improve
the speed with which goal-directed questions are identified with no
impact on agent performance.

Our presentation includes a discussion of Incomplete STRIPS, belief maintenance
and planning in incomplete domains, strategies for KA, an empirical evaluation in several domains,
 and a conclusion.

\section{Background \& Representation}\label{sec:background}

Incomplete STRIPS relaxes the classical STRIPS model to allow for
possible preconditions and effects \citep{Garland02}.  
Incomplete STRIPS domains are identical to STRIPS domains, with the exception
that the actions are incompletely specified.  Much like planning with incomplete
state information \citep{bonet00planning}, the action incompleteness is not
completely unbounded.  The preconditions and effects of each action can be any
subset of the propositions $P$; the incompleteness is with regard to a lack of
knowledge about which of the subsets correspond to each precondition and effect.        

\und{Incomplete STRIPS Domains} An incomplete STRIPS domain ${D}$  defines the
tuple ($P$, ${A}$, $I$, $G$, $F$), where: $P$ is a  set of propositions, ${A}$
is a set of incomplete  action descriptions, $I \subseteq P$ defines a set of
initially true  propositions, $G \subseteq P$ defines the goal propositions, and
$F$ is a set of propositions describing incomplete domain features. Each action
${a} \in {A}$ defines $pre({a}) \subseteq P$, a set of known preconditions,
$add({a}) \subseteq P$, a set of known add  effects, and $del({a}) \subseteq P$,
a set of known delete effects.   The set of incomplete domain features $F$ is
comprised of propositions of the form $pre({a}, p)$, $add({a}, p)$, and
$del({a}, p)$, each indicating that $p$ is a respective possible precondition, add effect, or
delete effect of $a$.

Consider the following incomplete domain: $P = \{p, q, r, g\}$, ${A} =
\{{a}, {b}, {c}\}$, $I = \{p, q\}$,  $G= \{g\}$, and $F = \{pre(a, r), add(a, r),
del(a, p),del(b, q),pre(c, q) \}$.  The
known features of the actions are defined:  $pre({a}) = \{p, q\}$, $  pre({b}) =
p\},  add({b}) = \{r\}$, and  $ pre({c}) =
\{r\},  add({c}) = \{g\}$. 

An interpretation $F^i \subseteq F$ of the incomplete STRIPS domain
defines a STRIPS domain, in that every feature $f \in F^i$
indicates that a possible precondition or effect is a respective known
precondition or known effect; those features not in $F^i$ are
not preconditions or effects.   



\und{Incomplete STRIPS Plans} A plan $\pi$ for ${D}$ is a sequence of
actions, that when applied, {\em can lead} to a state where the goal is
satisfied.  A plan $\pi = ({a}_0,  ..., {a}_{n-1})$ in an incomplete
domain ${D}$ is a sequence of  actions, that corresponds to the {\em
optimistic} sequence of states $(s_0, ...,  s_n)$, where $s_0 = I$,
$pre({a}_t) \subseteq s_t$ for $t = 0,...,  n$, $G \subseteq s_n$,
and $s_{t+1} = s_t \backslash del({a}_t)  \cup
add({a}_t) \cup \{p | add(a, p) \in
F\}$ for $t = 0,...,
n-1$.

For example, the plan $({a}, {b}, {c})$ corresponds to the
state sequence $(s_0 = \{p, q\}, s_1 = \{p, q, r\}, s_2 = \{q, r\}, s_3 = \{q,
r, g\})$, where the goal is satisfied in $s_3$.  We note that $r \in s_1$ even
though $r$ is only a possible add effect of $a$; without listing $r$ in $s_1$,
the known precondition of $b$ would not be satisfied.  While it is possible that
in the true domain $r$ is not an add effect of $a$, in the absence of contrary
information we optimistically assume $r$ is an add effect so that we can
synthesize a plan.   Pessimistically disallowing such plans is admissible, but
constraining, and we prefer to find a plan that may work to finding no plan at
all.  Naturally, we prefer plans that succeed under more interpretations.

\section{Belief Maintenance \& Planning}

An agent can act, ask questions, and plan.  Acting and asking a question provide
observations of the incomplete domain, and planning involves predicting future
states (in the absence of observations).  In the following, we discuss how
observations can be filtered to update an agent's knowledge $\phi$ (defined over
the literals of $F$), and what can be assumed about predicted states
(when taking knowledge into account).  We denote by $d(\pi)$ a plan's failure
explanations/diagnoses, which is represented by a propositional sentence over
$F$.

We use $\phi$ to reason about actions and
plans by making queries of the form $\phi \models
add({a}, p)$ (Is $p$ a known add effect of
${a}$?), $\phi \not\models add({a}, p)$ and $\phi
\not \models \neg add({a}, p)$  (Is $p$ a
possible/unknown add effect of ${a}$?),  or $\phi \models d(\pi)$ (Is the
current knowledge consistent with every interpretation where $\pi$ is guaranteed
to fail?).  It is often the case that it is unknown if an incomplete feature
${\sf f} \in F$ exists in the true domain that is consistent with $\phi$ (i.e., $\phi \not\models f$ and $\phi \not\models
\neg f$), and we denote this by ``$\phi?f$''.



\und{Filtering Observations} An agent that acts in incomplete STRIPS domains
will start with no knowledge of the incomplete features (i.e., $\phi = \top$),
however, taking actions provides state transition observations of the form $o(s,
a, s')$, and asking questions (i.e., ``Is $f$ true or false?'') provides
observations of the form $f$ or $\neg f$.  Thus the function ${\tt filter}$
returns the updated knowledge $\phi'$ after an observation, and is
defined:
\begin{align*}
{\tt filter}(\phi, f) & = \phi \wedge f \\
{\tt filter}(\phi, \neg f) & = \phi \wedge \neg f \\
{\tt filter}(\phi, o(s, a, s)) & = \phi \wedge ((fail \wedge o^-) \vee  o^+)\\
{\tt filter}(\phi, o(s, a, s')) & = \phi \wedge  o^+
\end{align*}
 where 
 
\begin{eqnarray*}
o^- &=& \bigvee\limits_{\substack{pre({a},p) \in 
F:\\ p \not\in s} } pre({a},p) 
\label{eqn:update1} \\
o^+ &=& o^{pre} \wedge o^{add} \wedge o^{del}\label{eqn:update2}\\
o^{pre} &=& \bigwedge\limits_{\substack{pre({a},p)
 \in F:\\ p \not\in s} }\neg pre({a},p)
 \label{eqn:update3}  \\
o^{add} &=& \bigwedge\limits_{\substack{add({a},p) \in 
F:\\ p\in s'\backslash s} }add({a},p) 
  \wedge \bigwedge\limits_{\substack{add({a},p) \in 
  F: \\ p \not\in  s\cup s'}} \neg 
  add({a},p)   \label{eqn:update4}\\
o^{del} &=& 
\bigwedge\limits_{\substack{del({a},p) \in {\sf
F}: \\p \in s \backslash s'}}
del({a},p)  \wedge
\bigwedge\limits_{\substack{del({a},p)
\in F:\\ p \in s\cap s'}}\neg
del({a},p)  \label{eqn:update5}
\end{eqnarray*}

\noindent We assume that the state will remain unchanged upon executing
an action whose precondition is not satisfied, and because the
state is observable, ${\tt filter}(\phi, o(s, a, s))$ references the case where
the state does not change and ${\tt filter}(\phi, o(s, a, s'))$, the case where
it changes.  If the state does not change, then either the action failed
($o^-$) and one of its unsatisfied possible preconditions is a precondition  or
the action succeeded ($o^+$).  We use the $fail$ proposition to denote
interpretations under which a plan failed because it is not always observable
that the plan has failed. If the state changes, then the agent knows that the
action succeeded. If an action  succeeds, the agent can conclude that i) each possible precondition that was not satisfied is not a precondition ($o^{pre}$), ii) each possible add effect that
appears in the successor but not the predecessor state is an add effect and each that does not appear in either state is not an add 
effect ($o^{add}$), iii) each possible delete effect that appears in the
predecessor but not the successor is a delete effect and each that  appears in both states is
not ($o^{del}$).



\und{Planning} We label predicted state propositions and actions with domain
interpretations that will respectively fail to achieve the proposition or fail
to achieve the preconditions of an action.  That is, labels indicate the cases
where a proposition will be false (i.e., the plan fails to establish the
proposition). Labels $d(\cdot)$ are represented as  propositional sentences over
$F$ whose models correspond to failed domain interpretations.

Initially, each proposition $p_0 \in s_0$, in the state from which a plan is
generated, is labeled $d(p_0) = \perp$ to denote that there are no
interpretations in the current state where a proposition may be false (the state is fully-observable), and each $p_0 \not\in
s_0$ is labeled $d(p_0)=\top$ to denote they are known false.
For all $t \geq 0$, we define:

 \begin{align}
%\label{eqn:actlabel}
\notag d({a}_t) =&  
d(a_{t-1}) \vee\hspace*{-.5cm} \bigvee\limits_{\substack{p \in pre(a)
\;\text{or}\;\\\phi \models pre(a, p)}}\hspace*{-.5cm} d(p_t) \vee
\hspace*{-.5cm}\bigvee\limits_{\substack{p : \phi?pre(a, p)}}
\hspace*{-.5cm}(d(p_t)\wedge pre({a}_t, p)  ) \notag %& : { t $\geq$ 1 }
%\end{array}\right.
\\
\notag\hspace*{-.1cm}d(p_{t+1}) &= \left\{
\begin{array}{l@{\;}l@{\;}r}
d(p_t) \wedge d({a}_t) & : p \in add({a}_t) \\ 
 & \;\text{or}\; \phi \models add({a}_t, p)\\
d(p_t) \wedge (d({a}_t) \vee & : \phi?add({a}_t,p)\\
 \hspace*{1.25cm} \neg add({a}_t, p)) \\  
\top & : p \in del({a}_t)\\
 & \;\text{or}\; \phi \models del({a}_t,p)\\
d(p_t) \vee  del({a}_t, p)  &: \phi?del({a}_t, p)\\
d(p_t) & : {otherwise} 
\end{array}
\right. 
\end{align}

\noindent where $d({a}_{-1}) = \perp$. The intuition behind the label
propagation is that an action will fail  in the
domain interpretations $d({a}_t)$ where a prior action failed, a known
precondition is not satisfied, or a possible precondition is not satisfied. As defined for
$d(p_{t+1})$, the plan will fail to achieve a proposition at time $t+1$ in all
interpretations where i) the plan fails to achieve the proposition at time $t$ and the action fails, ii) the plan fails to achieve the proposition at
time $t$ and the action fails or it does not add the proposition in the
interpretation, iii) the action deletes the proposition, iv) the plan fails to
achieve the proposition at time $t$ or in the interpretation the action deletes
the proposition, or v) the action does not affect the proposition and  prior
failures  apply.             

A consequence of our definition of action failure is that each action fails if
any prior action fails.  This definition follows from the semantics that the
state becomes undefined if we apply an action whose preconditions are not
satisfied.  While we use this notion in plan synthesis, we explore the semantics
that the state does not change (i.e., it is defined) upon failure when acting in
incomplete domains.  The pragmatic reason that we define action failures in this
manner is that we can determine all failed interpretations affecting a plan $d(\pi)$,
by defining $d(\pi) = d({a}_{n-1}) \vee \bigvee_{p \in G} d(p_n)$ (i.e.,
failure to execute an action is propagated to a
failure to achieve the goal). For example, our plan example from the previous
section has the failure explanation label $d(\pi) = 
pre(a, r) \vee  del(a, p) \vee (del(b, q) \wedge pre(c, q))$.

\und{Incomplete Domain Relaxed Plans} The \default{} planner
\citep{bryce-icaps11} guides its expansion of plans that are labeled with
failure explanations by computing relaxed plans with failure explanations.  We
also use such relaxed plan failure expalantions to select questions.  Finding a
relaxed plan that attempts to minimize failure explanations involves propagating
failed interpretation labels in a planning graph.  Propagating labels relies on
selecting an action to support each proposition, and we select the supporter
$a_{t+k}(p)$ at step $k$ of the planning graph for state $s_t$ with the
fewest failed interpretations, denoted by its label $\hat{d}(a_{t+k}(p))$. 

A relaxed planning graph with propagated labels is a layered graph of sets of
vertices of the form $({\cal P}_t, {\cal A}_t, ..., {\cal A}_{t+m},
{\cal P}_{t+m+1})$. The relaxed planning graph built for a state
${s}_t$ defines ${\cal P}_0 = \{{p}_t | p \in {s}_t\}$,
${\cal A}_{t+k} = \{ {a}_{t+k} | \forall_{p \in {pre}({a})}
{p}_{t+k} \in {\cal P}_{t+k}, {a} \in {A} \cup A(P)\}$, and
${\cal P}_{t+k+1} = \{p_{t+k+1} | {a}_{t+k} \in {\cal A}_{t+k}, p
\in {add}({a}) \cup \{p | \phi\not\models\neg add({a}, p)\}$, for $k
= 0, ..., m$.  Much like the successor function used to compute next states, the
relaxed planning graph assumes an optimistic semantics for action effects by
adding possible add effects to proposition layers, but, as we will explain
below, it associates failed interpretations with the possible adds.

 Each planning graph vertex has a label, denoted $\hat{d}(\cdot)$.  The failed
 interpretations $\hat{d}(p_t) $ affecting a proposition are defined such that
 $\hat{d}(p_t) = d(p_t)$, and for $k \geq 0$,
\begin{align}
\hat{d}({a}_{t+k}) &= \hspace*{-.2cm} 
\bigvee\limits_{\substack{p \in pre({a}) \;\text{or}\;\\ \phi\models
pre(a, p) }} \hspace*{-.2cm}
\hat{d}(p_{t+k}) \vee \hspace*{-.25cm} \bigvee\limits_{\phi?pre(a, p)}
\hspace*{-.25cm} (\hat{d}(p_{t+k})  \wedge pre({a}, p) )\notag\\ \notag
\hspace*{-.6cm} \hat{d}(p_{t+k+1}) &= \left\{\begin{array}{l@{\;}l@{\;}r} \hat{d}(\hat{a}_{t+k}(p)) & : p \in add(\hat{a}_{t+k}(p))\\
 & \;\text{or}\; \phi\models
add(\hat{a}_{t+k}(p), p)\\ 
\hat{d}(\hat{a}_{t+k}(p)) \vee& : \phi?add(\hat{a}_{t+k}(p), p)\\
\neg add(\hat{a}_{t+k}(p), p)
\end{array}\right. \hspace*{-.1cm}\label{eqn:hprop}
\end{align}
% Propositions in the planning graph initially have the same faults associated
% with them as in state $\tilde{s}_t$ and are defined by $d(\cdot)$.
Every action in every level $k$ of the planning graph will fail in any
interpretation where their preconditions are not supported.  A proposition will
fail  to be achieved in any interpretation where the chosen supporting action
fails to add the proposition.

% We note that the rules for propagating labels in the planning graph differ from
% the rules for propagating labels in the state space.  In the state space, the
% action failure labels include interpretations where any prior action fails.  In
% the relaxed planning problem, the action failure labels include only
% interpretations affecting the action's preconditions, and not prior actions; it
% is not clear which actions will be executed prior to achieving a proposition
% because many actions may be used to achieve other propositions at the same time
% step.

%\und{Choosing Supporters} While only a single action or noop may be required to support a proposition, using multiple supporters can increase the number of incomplete domain interpretations that support it (by ensuring that not all sources of support are subject to the same faults).  We wish to select a set $\hat{\cal S}_{t+k}(p) \subseteq \{\tilde{a} \in \hat{\cal A}_{t+k} | p \in \text{add}(\tilde{a}) \cup \widetilde{\text{add}}(\tilde{a})\}$ to define a most preferred $\hat{d}_{t+k+1}(p)$ (i.e., have fewer explanations of failure).  We use a greedy algorithm to incrementally add actions to $\hat{\cal S}_{t+k}(p)$, and check if $\hat{d}_{t+k+1}(p)$ is improved.  
%
%The greedy algorithm proceeds as follows.  We consider all singleton sets of actions, and select the most preferred (having the fewest failure explanation models or prime implicants, and breaking ties by selecting noop actions).  To the most preferred action set, we add each action and determine the most preferred, two element action set.  If none of the two element action sets are more preferred than the best single action set, we define $\hat{\cal S}_{t+k}(p)$ as the best single action set.  Otherwise, the algorithm continues to extend the best action set with one action at a time until it cannot find a more preferred action set (by counting models or prime implicants).
%
% alternative definitions (one for each possible supporter) of the set $\hat{\cal S}_{t+k}(p)$, and select the supporter that has the most preferred definition of  $\hat{d}_{t+k+1}(p)$ (breaking ties by first preferring noop actions and second preferring actions whose first appearance in the planning graph is earliest).  With a single element in $\hat{\cal S}_{t+k}(p)$, we evaluate all single element extensions of the set (making it a two element set), choosing the extension that most improves the measure of $\hat{d}_{t+k+1}(p)$ (depending on the chosen measure, either reducing the number of models, or prime implicants); if no extension improves the measure, then the algorithm terminates and returns $\hat{\cal S}_{t+k}(p)$ with a single element.  In our implementation, we allow at most two supporters per proposition, but it is possible to allow an arbitrary number of supporters in this fashion by extending the set of supporters greedily until no new extension improves its measure.
%



%\und{Heuristic Computation}  
 The relaxed planning graph expansion terminates
at the level $t+k+1$ where the goals have
been reached at $t+k+1$. The $h^{\sim FF}$ heuristic makes use of the chosen
supporting action $\hat{a}_{t+k}(p)$ for each proposition that requires support in the relaxed
plan, and, hence, measures the number of actions used while attempting to
minimize failed interpretations.  The failure explanation of the relaxed plan is
defined by $d(\hat{\pi}) = \bigvee\limits_{p \in G} \hat{d}(p_{t+m+1})$.


\section{Goal-Directed Knowledge Acquisition}

We describe several approaches to formulating questions for a domain expert in
incomplete domains, which include asking all possible questions in an uninformed
manner, asking no questions but learning by trial and error, and using plans to
select goal directed questions. We discuss the approaches by increasing
sophistication and follow with an empirical evaluation.

\und{Uninformed QA} Uninformed QA (UQA) is the strategy taken by an agent that
cannot or will not plan or act under uncertainty and thus must ask the domain expert about each incomplete
feature of the domain without regard to its relevance to goal achievement.  As
such, uninformed QA based agents will ask a set of questions $Q_{F} = F$ about
every feature in $F$, formulate a classical plan, and execute the plan.

\und{Non-QA} Non-QA is the strategy taken by an agent that would rather act
under uncertainty and ask no questions of the domain expert.  Non-QA agents are
potentially reckless because learning about the domain sometimes requires that
they apply actions whose preconditions may be unsatisfied.  

Using $\phi$, it is possible to determine if the next action in a plan, or any
subsequent action, can or will fail.  If  $\phi \wedge d(\pi)$ is
satisfiable, then $\pi$ {\em can} fail, and if $\phi \models d(\pi)$,
then $\pi$ {\em will}  fail.  
% The agent will execute an action if it may not
% fail, but will re-plan if any action is guaranteed to fail.  If
% \goalie{} determines that its next action will fail, or a prior action failed ($\phi \models fail$), then
% it will re-plan.  \goalie{}  uses $\phi$ to modify the actions during
% re-planning by checking for each incomplete domain feature $f \in {\sf
% F}({D})$ if $\phi \models f$ or if $\phi \models \neg f$.  Each such
% literal entailed by $\phi$ indicates if the respective action has the possible
% feature as a known or impossible feature; all other features remain as possible
% features. 
Algorithm \ref{alg:replan} is the strategy used by the Non-QA agent.  The
algorithm involves initializing the agent's knowledge and plan (line 1), and then while
the plan is non-empty and the goal is not achieved (line 2) the agent proceeds
as follows.  The agent selects the next action in the plan (line 3) and
determines if it can apply the action (line 4).  If it applies the action, then
the next state is returned by the environment/simulator (line 5) and the agent
updates its knowledge (line 6) and state (line 7),
otherwise the agent determines that the plan will fail (line 9).  If the plan
has failed (line 11), then the agent forgets its knowledge of the plan failure
by projecting over $fail$ (line 12) and finds a new plan using its new knowledge
(line 13).

For example, the non-QA agent might observe the state transition $o_1
= o(\{p,q\}, a, \{p,q\})$ upon executing $a$, and $\phi' = {\tt filter}(\phi,
o_1) = \neg del(a, r)$.  The agent must re-plan because $\phi' \models d(\pi)$.

% \goalie{} is not hesitant to apply actions that may fail because trying actions
% is its only way to learn about them.  However, \goalie{} is able to determine
% when actions will fail and re-plans.  More conservative strategies are possible
% if we assume that \goalie{} can query a knowledge engineer about action features
% to avoid potential plan failure, but we leave such goal-directed knowledge
% acquisition for future work.

%\begin{algorithm}[t]
%\SetLine
%\KwIn{state $s$, goal $G$, actions ${A}$}

\begin{algorithm}[t]
\caption{Non-QA$(s, G, {A})$}
\label{alg:replan}
  \begin{algorithmic}[1]
\STATE $\phi \leftarrow \top$; $\pi \leftarrow {\tt plan}(s, G, {A}, \phi)$;
\WHILE{$\pi \not= ()$ and $G\not\subseteq s$}
\STATE $a \leftarrow \pi.first()$; $\pi \leftarrow \pi.rest()$;
\IF{${pre}(a) \subseteq s$ and
$\phi \not\models d(\pi)$}
\STATE $s ' \leftarrow Execute(a)$; 
\STATE $\phi \leftarrow {\tt filter}(\phi, o(s, a, s'))$; 
\STATE $s \leftarrow s'$; 
\ELSE 
\STATE  $\phi \leftarrow \phi \wedge fail$; 
 \ENDIF
 \IF{$\phi \models fail$ }
\STATE  	 $\phi \leftarrow \exists_{fail}  \phi$;
\STATE  	 $\pi \leftarrow {\tt plan}(s, G, {A}, \phi)$;
 \ENDIF
\ENDWHILE
\end{algorithmic}
\end{algorithm}


\und{Goal Directed QA}Goal Directed QA agents plan under uncertainty, similar to
the Non-QA agent, but ask about action features that are relevant to the plan.
There are a number of strategies for using a plan to generate questions, and we assume
that the goal directed QA agent continues to ask questions and re-plan until it
finds a plan that is guaranteed to succeed. The strategies include: asking about
all incomplete features related to actions in a plan, asking about only those
features that can cause failure, and ranking the questions based on the
diagnoses of plan failure to ``fail fast''.

\und{Plan Relevant Questions} Without computing failure explanations, it is
possible to determine relevant questions by inspecting the actions in the plan
and their possible preconditions and effects so that the set of questions is
defined:
\begin{align}
Q_{\pi} =&  \{pre(a_t, p) | \phi?pre(a_t, p), a_t \in \pi\}
\cup\notag\\ &\{add(a_t, p)| \phi?add(a_t, p), a_t \in \pi\} \cup\notag\\ 
&\{del(a_t,p)|\phi?del(a_t, p), a_t \in \pi\}\notag
\end{align}

By using $Q_{\pi}$ and no plan failure explanation, an agent is relegated to
asking each question because it cannot determine if the plan will fail given the
answers it has received thus far.  The agent can only re-plan afterward and
determine if the plan contains actions with incomplete features.

The example plan would lead to $Q_{\pi} = F$ because incomplete feature in
$F$ relates to an action in the plan.  However, as we saw in $d(\pi)$, $add(a,
r)$ is irrelevant because the plan will succeed no matter its value.

\und{Plan Failure Diagnosis Relevant}  A question is relevant to a plan
${\pi}$ if the incomplete feature ${\sf f}$ is entailed by a potential
diagnosis $\delta$ of plan failure.  Each diagnosis $\delta$ of the plan
failure explanation $d({\pi})$ 
is a conjunction of incomplete features that must interact to destroy the plan. 
Thus, if $\delta \models d({\pi})$ and $\delta \models {\sf f}$, then:  
\begin{align}
Q_{d(\pi)} =&  \{{\sf f} | \delta \models d(\pi)
, \delta \models {\sf f} \;\text{or}\; \delta \models \neg{\sf f}\}\notag
\end{align}

The example plan defines $Q_{d(\pi)} = \{pre(a, r), del(a, p), del(b, q),
pre(c, q) \}$ because each feature appears in a diagnosis.

\und{Ranking Relevant Questions}  
The features in smaller cardinality diagnoses have more impact on the plan
because a smaller number of unfavorable answers are needed to prove the plan
will fail; asking about these features will enable an agent to fail fast. 
Moreover, features appearing in more diagnoses have a high impact on plan
failure.  Using this {\em diagnosis-impact} (DI) measure, we can prioritize  by selecting the  ${\sf f}$ where 
% feature:
\begin{align}
{\sf f} = \underset{{\sf f}  \in Q_{d(\pi)}}{\operatorname{argmax}}
\sum\limits_{\substack{\delta : \delta \models d(\pi),\\ \delta\models {\sf
f}}}
\frac{1}{|\{{\sf f}| \delta \models {\sf f}\}|^2}
\notag 
\end{align}
The denominator of the expression above is squared to penalize the contribution
of larger diagnoses.  DI determines the incomplete feature most
likely to cause the plan to fail.

Using DI for the example plan questions will select $pre(a, r)$ and $del(a,
p)$ as equally preferred questions because both appear in a size one diagnosis.


An alternative, based on  {\em Shannon entropy} (SE),
selects the question ${\sf f}$ where
\begin{align}
{\sf f} = &\underset{{\sf f}  \in Q_{d(\pi)}}{\operatorname{argmin}} -
\sum\limits_{{\sf f} \in \{{\sf f}, \neg {\sf f}\}} p_{\sf f} \log p_{\sf f} 
\notag\\
 p_{\sf f} =& |M({\tt filter}(\phi, {\sf f}) \wedge 
d(\pi))|/2^{|F|}\notag
\end{align}
and $M(\cdot)$ denotes set of propositional models of some sentence.
The probability $p_{\sf f}$ that the plan fails under the true domain model
is the proportion of propositional models where after filtering ${\sf f}$ the
plan will fail.  If it is the case that ${\tt filter}(\phi, {\sf f}) \models 
d(\pi)$, then $\pi$ will  fail and the agent must re-plan; when computing
$p_{\sf f}$ and the plan fails, we compute a new plan $\pi'$ given the knowledge
$\phi' = {\tt filter}(\phi, {\sf f})$ and replace $d(\pi)$ with $d(\pi')$.  If
no plan $\pi'$ exists, we define $p_{\sf f} = 1$ to denote that the absence of
a plan indicates failure. 

For example, both $pre(a, r)$ and $del(a,
p)$ as equally preferred questions because their entropy is 0.16, whereas both
$del(b, q)$ and $pre(c, q)$ have entropy 0.32.



\section{Empirical Evaluation}
The empirical evaluation is divided into three sections:  the domains used for
the experiments, the test setup used, and results.  The questions that we would like to answer
include:



\begin{packed_itemize}
  \item Q1: Does reasoning about incompleteness during planning 
  effect KA strategy performance?
  \item Q2: Which KA strategy has best solution time and
  number of questions trade-off?
  \item Q3: Do relaxed plans reduce the cost of KA?
\end{packed_itemize}



\begin{table}[t]\small\centering\begin{tabular}{|l|cc|}\hline
Domain & Non-QA  & SE \\ \hline
Br	0.25	 & 1.0/1.2/6.4/-	 & 1.0/1.0/23.4/3.8 	\\ \hline
Br	0.5	 & 0.9/1.2/4.5/-	 & 1.1/1.0/4.4/1.7 	\\ \hline
Br	0.75	 & 1.0/1.0/3.9/-	 & 1.2/1.0/1.3/1.2 	\\ \hline
Br	1.0	 & 1.1/1.0/1.3/-	 & 1.3/1.0/2.4/1.2 	\\ \hline
\hline
Pw	0.25	 & 1.0/1.0/1.7/-	 & 1.3/0.9/3.7/1.6 	\\ \hline
Pw	0.5	 & 0.5/1.1/1.6/-	 & 1.5/1.1/6.5/1.7 	\\ \hline
Pw	0.75	 & 0.7/1.1/1.8/-	 & 1.2/1.0/1.6/1.1 	\\ \hline
Pw	1.0	 & 0.8/0.9/0.8/-	 & 1.1/1.0/1.1/1.1 	\\ \hline
\hline
Pp	0.25	 & 0.8/1.0/2.3/-	 & 1.1/1.0/2.2/1.1 	\\ \hline
Pp	0.5	 & 2.3/1.0/3.6/-	 & 1.4/1.0/0.9/0.9 	\\ \hline
Pp	0.75	 & -/-/-/- 	 & 2.1/1.0/1.0/1.0 	\\ \hline
Pp	1.0	 & -/-/-/- 	 & 1.9/1.1/0.6/0.8 	\\ \hline
\hline
Bw	0.25	 & 0.8/1.2/9.7/- 	 & 1.0/0.9/18.5/3.3 	\\ \hline
Bw	0.5	 & 0.7/1.2/12.2/-	 & 1.0/1.0/18.9/2.3 	\\ \hline
Bw	0.75	 & 0.7/1.0/11.5/-	 & 1.2/0.9/8.8/1.4 	\\ \hline
Bw	1.0	 & 0.5/0.9/27.3/-	 & 1.5/0.9/14.0/1.3 	\\ \hline
\end{tabular}\caption{\label{tab:plannerComp} FF/\default{}
 Performance Ratio \\(Num Solved/Num Steps/Time/Questions).}\end{table}


\und{Domains} We use four domains in the evaluation: a modified Pathways,
Bridges,  a modified PARC Printer, and Barter World \citep{bryce-icaps11}.  In
all domains, we derived multiple instances by randomly (with probabilities 0.25,
0.5, 0.75, and 1.0 for each action) injecting incomplete  features.   
With these variations of the domains, the instances include up to ten thousand
incomplete  features each. All results are taken from ten random instances
(varying $F$) of each problem and three ground-truth domains selected by the
simulator.
% The problem instances and generators are available at \href{}{\it withheld for
% blind review}.

The Pathways (Pw) domain from the International Planning Competition  (IPC) involves actions that model chemical reactions in signal
transduction pathways.  Pathways is a naturally incomplete domain where the lack
of knowledge of the reactions is quite common because they are an active
research topic in biology.  

The Bridges (Br) domain consists of a traversable grid and the task is to find a
different treasure at each corner of the grid. In Bridges,
a bridge might be required  to cross between some grid locations (a possible
precondition), many of the bridges may have a troll living
underneath that will take all the treasure accumulated (a possible delete
effect), and the corners may give additional treasures (possible add
effects).  Grids are square and vary in dimension (2-16).

The PARC Printer (Pp) domain from the IPC involves planning paths for sheets of
paper through a modular printer.  A source of domain incompleteness is that a module
accepts only certain paper sizes, but its documentation is incomplete.  Thus,
paper size becomes a possible precondition to actions using the module.

The Barter World (Bw) domain involves navigating a grid and bartering items to
travel between locations.  The domain is incomplete because actions that acquire
 items are not always known to be successful (possible add effects) and traveling between locations may require
certain items (possible preconditions) and may result in item loss
(possible delete effects). Grids vary in dimension (2-16) and items
in number (1-4).



\begin{table}[t]\small\centering\begin{tabular}{|l|cc|}\hline
Domain & Non-QA  &  $Q_F$ \\ \hline
Br	0.25	&271/22.1/1.6/- 	&{\bf 274}/{\bf 20.2}/{\bf 0.5}/57.5 	\\ \hline
Br	0.5	&199/27.1/3.0/- 	&{\bf 251}/{\bf 17.9}/{\bf 0.4}/114.1 	\\ \hline
Br	0.75	&131/18.2/1.9/- 	&{\bf 200}/{\bf 13.5}/{\bf 0.2}/65.7 	\\ \hline
Br	1.0	&89/{\bf 12.0}/0.9/- 	&{\bf 190}/12.1/{\bf 0.2}/107.8 	\\ \hline
\hline
Pw	0.25	&91/{\bf 13.5}/0.4/- 	&{\bf 121}/16.8/{\bf 0.3}/28.9 	\\ \hline
Pw	0.5	&80/{\bf 21.4}/4.5/- 	&{\bf 180}/27.2/{\bf 0.5}/75.8 	\\ \hline
Pw	0.75	&70/{\bf 12.4}/0.5/- 	&{\bf 130}/15.2/{\bf 0.3}/73.2 	\\ \hline
Pw	1.0	&40/{\bf 11.2}/0.5/- 	&{\bf 127}/14.5/{\bf 0.3}/104.3 	\\ \hline
\hline
Pp	0.25	&37/{\bf 9.5}/{\bf 0.3}/- 	&{\bf 151}/11.0/0.4/36.4 	\\ \hline
Pp	0.5	&6/{\bf 9.3}/{\bf 0.4}/- 	&{\bf 151}/11.0/0.4/75.3 	\\ \hline
Pp	0.75	&-/-/-/-  	&{\bf 150}/{\bf 11.0}/{\bf 0.4}/109.4 	\\ \hline
Pp	1.0	&-/-/-/-  	&{\bf 150}/{\bf 11.0}/{\bf 0.4}/145.2 	\\ \hline
\hline
Bw	0.25	&276/13.6/4.7/- 	&{\bf 321}/{\bf 11.5}/{\bf 0.8}/232.8 	\\ \hline
Bw	0.5	&179/11.0/7.7/- 	&{\bf 239}/{\bf 9.2}/{\bf 0.4}/205.7 	\\ \hline
Bw	0.75	&118/12.1/2.6/- 	&{\bf 207}/{\bf 8.1}/{\bf 0.3}/172.8 	\\ \hline
Bw	1.0	&11/12.5/3.6/- 	&{\bf 20}/{\bf 8.6}/{\bf 0.3}/189.8 	\\ \hline
\end{tabular}\caption{\label{tab:questionComp2} Extreme Strategy Average
Performance. Bold indicates best performance.  (Num Solved/Num
Steps/Time (s)/Questions).}\end{table}
 


\und{Test Setup} The tests were run on a Linux machine with a 3 Ghz processor,
with a  2GB memory limit and 60 minutes time limit for each instance. All code
was written in Java and run on the 1.6 JVM. The \default{} planner uses a greedy
best first search with deferred heuristic evaluation and a dual-queue for
preferred and non-preferred operators \citep{DBLP:journals/jair/Helmert06}. 
%Both planners also use the same planning
% graph implementation.

\und{Results} To answer Q1, Table \ref{tab:plannerComp} compares two planners
and two KA strategies.  The entries in the table are the ratio of
the number of problems solved, number of steps taken by the agent, total time,
and questions asked when using a planner that ignores incompleteness
(\default{} using the FF heuristic \citep{hoffmann:nebel:jair-01}) and a planner
that plans with incompleteness (\default{}).  The columns list results for the
Non-QA agent and the SE strategy.  We see that total time and number of steps is
reduced in the Non-QA agent when planning with incompleteness (as also found by
\citet{bryce-icaps11}), and that the total time and number of questions is
reduced when using SE.  The
results demonstrate that planning with incompleteness is beneficial independent
of asking questions.

\begin{table*}[t]
\begin{minipage}{4.96in}
\small\begin{tabular}{|@{}l@{}|@{}c@{ }c@{ }c@{ }c@{}|}\hline
Domain & $Q_{\pi}$  &$Q_{d(\pi)}$ & DI & SE \\ \hline
Br	0.25	&{\bf 272}/19.8/1.8/4.1 	&{\bf 272}/{\bf 19.8}/{\bf 1.8}/2.8 	&{\bf 272}/19.9/1.9/{\bf 2.2} 	&271/20.0/15.4/2.3 	\\ \hline
Br	0.5	&{\bf 232}/17.0/{\bf 3.1}/18.2 	&231/17.1/3.7/12.4 	&226/16.8/4.4/{\bf 8.9} 	&219/{\bf 16.1}/146.3/10.3 	\\ \hline
Br	0.75	&{\bf 191}/12.3/1.9/24.5 	&{\bf 191}/12.3/{\bf 1.9}/17.5 	&183/12.0/3.1/13.2 	&171/{\bf 11.2}/75.9/{\bf 12.6} 	\\ \hline
Br	1.0	&172/11.0/3.6/35.0 	&169/10.9/{\bf 3.4}/22.1 	&{\bf 173}/10.7/3.5/17.1 	&151/{\bf 9.5}/128.9/{\bf 15.9} 	\\ \hline
\hline
Pw	0.25	&{\bf 91}/{\bf 13.3}/0.5/1.3 	&{\bf 91}/{\bf 13.3}/0.4/{\bf 1.0} 	&{\bf 91}/{\bf 13.3}/{\bf 0.3}/{\bf 1.0} 	&{\bf 91}/{\bf 13.3}/1.1/{\bf 1.0} 	\\ \hline
Pw	0.5	&130/27.5/10.3/10.9 	&130/27.5/{\bf 10.2}/7.7 	&{\bf 141}/24.8/11.5/5.0 	&101/{\bf 16.4}/42.0/{\bf 3.0} 	\\ \hline
Pw	0.75	&{\bf 130}/15.8/1.8/11.9 	&{\bf 130}/16.3/1.8/7.2 	&110/{\bf 14.5}/{\bf 1.6}/{\bf 5.6} 	&110/14.7/54.5/5.9 	\\ \hline
Pw	1.0	&{\bf 127}/14.6/{\bf 1.5}/21.7 	&{\bf 127}/15.2/2.1/12.8 	&117/14.5/2.6/{\bf 10.2} 	&118/{\bf 14.2}/83.6/10.8 	\\ \hline
\hline
Pp	0.25	&{\bf 138}/10.6/1.4/3.5 	&{\bf 138}/10.6/1.3/3.5 	&136/{\bf 10.5}/{\bf 1.0}/{\bf 2.9} 	&136/{\bf 10.5}/6.1/3.0 	\\ \hline
Pp	0.5	&{\bf 126}/10.3/1.9/7.9 	&{\bf 126}/10.3/{\bf 1.8}/7.9 	&121/10.1/1.8/{\bf 6.0} 	&110/{\bf 9.8}/24.3/6.4 	\\ \hline
Pp	0.75	&{\bf 83}/9.5/2.3/14.8 	&{\bf 83}/9.5/2.0/14.8 	&79/9.4/{\bf 2.0}/{\bf 11.1} 	&71/{\bf 9.1}/29.9/11.8 	\\ \hline
Pp	1.0	&{\bf 84}/9.1/2.9/21.3 	&83/9.1/{\bf 2.7}/21.1 	&80/{\bf 8.4}/3.8/{\bf 17.6} 	&78/8.4/36.5/18.1 	\\ \hline
\hline
Bw	0.25	&{\bf 321}/{\bf 11.9}/10.0/10.9 	&{\bf 321}/12.0/{\bf 9.5}/6.5 	&318/12.6/10.8/{\bf 3.7} 	&320/12.6/116.0/4.9 	\\ \hline
Bw	0.5	&{\bf 230}/{\bf 8.9}/{\bf 11.3}/20.7 	&{\bf 230}/9.2/12.2/15.2 	&227/9.6/15.6/{\bf 9.5} 	&228/9.5/171.1/11.5 	\\ \hline
Bw	0.75	&189/{\bf 7.4}/{\bf 3.2}/31.7 	&{\bf 190}/7.6/4.0/23.9 	&184/8.4/16.3/14.6 	&178/7.9/83.2/{\bf 14.5} 	\\ \hline
Bw	1.0	&13/{\bf 5.9}/{\bf 2.1}/28.5 	&14/6.5/3.0/25.6 	&{\bf 15}/7.1/7.9/{\bf 17.5} 	&13/6.3/201.9/20.2 	\\ \hline
\end{tabular}\caption{\label{tab:questionComp1} Goal-directed KA
Average Performance using \default{}. Bold indicates best performance. (Num
Solved/Num Steps/Time (s)/Questions)}
\end{minipage}
\begin{minipage}{1.75in}\centering
\small\begin{tabular}{|@{}l@{}|@{}c@{}|}\hline
Domain & SE  \\ \hline
Br	0.25	&1/1.0/8.3/1.0 	\\ \hline
Br	0.5	&1/1.0/33.1/1.1 	\\ \hline
Br	0.75	&1/1.0/51.1/1.1 	\\ \hline
Br	1.0	&1/1.0/55.3/1.1 	\\ \hline
\hline
Pw	0.25	&1/1.0/5.4/1.0 	\\ \hline
Pw	0.5	&1/1.0/6.0/1.0 	\\ \hline
Pw	0.75	&1/1.0/7.4/1.0 	\\ \hline
Pw	1.0	&1/1.0/18.5/1.0 	\\ \hline
\hline
Pp	0.25	&1/1.0/4.9/1.0 	\\ \hline
Pp	0.5	&1/1.0/8.8/1.1 	\\ \hline
Pp	0.75	&1/1.0/9.7/1.1 	\\ \hline
Pp	1.0	&1/1.0/8.8/1.0 	\\ \hline
\hline
Bw	0.25	&1/1.0/14.4/1.1 	\\ \hline
Bw	0.5	&1/1.0/15.0/1.1 	\\ \hline
Bw	0.75	&1/1.0/20.4/1.1 	\\ \hline
Bw	1.0	&1/1.0/18.1/0.9 	\\ \hline
\end{tabular}\caption{\label{tab:rpComp} Plan versus
Relaxed Plan Ratio in SE.}
\end{minipage}
\end{table*}

To answer Q2, Tables \ref{tab:questionComp2} and \ref{tab:questionComp1} list
the average performance of \default{} with various KA strategies.
We see that in comparing the two extremes, Non-QA or $Q_F$, $Q_F$ leads to more
solved instances and lower total time, but as expected a high number of
questions.  In comparing the approaches that use a plan to limit and/or bias
KA in Table \ref{tab:questionComp1}, we note the number of solved
instances is relatively similar, but the methods that prioritize questions based
on the plan failure explanations (DI or SE) ask the fewest questions.  The total
time taken is relatively similar among the methods, except for a noticeably higher average
time with $SE$ -- which is due to many hypothesized re-planning
episodes to compute the entropy.

To answer Q3, and follow up on the high cost of SE, Table \ref{tab:rpComp} lists
the performance ratios of using SE with actual plans versus relaxed plans when
re-planning is required.  We note that the performance is nearly identical, with
the exception of much lower total times -- which makes SE
competitive with DI.






% POND was run using ten planning graph samples per search node heuristic
% evaluation in a weighted (w=5) A* search, and  $\overline{\tau}$ equal to the
% minimum proportion of incomplete a ction instances that any other method
% satisfies the goal for the same problem.

%Those planners that solve more problems can be easily identified, and their overall relative plan quality and efficiency are evident by the cumulative plots.

%\und{\FFRISKY{} Implementation}The \FFRISKY{} planner is implemented in Java, and each of configurations of the planner share common source code, with the exception of their respective techniques for fault propagation in the state space and heuristic computation.  

% We use five configurations of the planner: \FFRISKY{}-$FF$, \FFRISKY{}-$PIk$ (k
% = 1, 2, 3), and \FFRISKY{}-$BDD$, that differ in how they reason about domain
% incompleteness.  \FFRISKY{}-$FF$ does not compute failure explanations and uses
% the FF heuristic;  it is inspired by the planner used by CA because it is likely
% to find a plan that will work for only the most optimistic domain
% interpretation.  \FFRISKY{}-$PIk$, where $k$ is the bound on the cardinality of
% the prime implicants, uses only prime implicants to compare failure
% explanations.  \FFRISKY{}-$BDD$ uses OBDDs to represent and count failure
% explanations.  The number of failed interpretations for a plan $\pi$ found by
% any of the planners is reported herein by counting models of an OBDD
% representing $d(\pi)$. The versions of the planner are compared by the
% proportion of interpretations of the incomplete domain that achieve the goal and
% total planning time in seconds. The plots in the following section depict these
% results using the cumulative percentage of successful domain interpretations and
% planning time to identify the performance over all problems and domains.  We
% also report detailed results on the number of solved problems per domain.
% 
% 
% \section{Related Work}
% 
% 
% Our investigation is an instantiation of model-lite planning \citep{modellite},
% and is motivated by work on instructable computing \citep{mable}. This work is a
% natural extension of the \citet{Garland02} model for evaluating plans in
% incomplete domains, but our method for computing plan failure explanations
% is slightly different, we actually synthesize plans in incomplete domains, and
% investigate question asking strategies.
% 
% Prior work of \citet{DBLP:conf/aips/ChangA06} addresses planning with incomplete
% models, but does not attempt to synthesize robust plans, which is similar to our
% planner that uses the FF heuristic.  We have shown that incorporating knowledge
% about domain incompleteness into the planner can lead to a more effective agent
% in both execution and question asking. We also differ in that we do not assume
% direct feedback from the environment about action failures, we can learn
% action preconditions, and we can query a domain expert.


\section{Conclusion}

We have found that reasoning about incompleteness planning 
leads to more effective trial-and-error agents and KA agents.  The plans help maintain a
goal-directed focus on knowledge acquisition and lower the overall strain upon a
domain expert to answer many questions.  We found that methods for prioritizing
questions are able to reduce the number of questions required to synthesize
successful plans and that relaxed plans  serve as a good substitute for
actual plans.  

\footnotesize
%\renewcommand{\baselinestretch}{.75}
\bibliography{aaai11QA,jared}
\bibliographystyle{icml2011}

\end{document}
